\newglossaryentry{poc}
{
  name=POC,
  description={Proof Of Concept : Veut démontrer un concept sans pour autant être achevé}
}
\newglossaryentry{si}
{
    name=SI,
    description={Un système d'information (SI) est un ensemble organisé de ressources qui permet de collecter, stocker, traiter et diffuser de l'information (Wikipédia)}
}
\newglossaryentry{open_source}
{
  name=open source,
  description={La désignation open source, ou « code source ouvert », s'applique aux logiciels dont la licence respecte des critères précisément établis par l'Open Source Initiative, c'est-à-dire les possibilités de libre redistribution, d'accès au code source et de création de travaux dérivés. (Wikipédia)}
}
\newglossaryentry{framework}
{
  name=framework,
  description={Un framework ou structure logicielle est un ensemble cohérent de composants logiciels structurels, qui sert à créer les fondations ainsi que les grandes lignes de tout ou d’une partie d'un logiciel (architecture). (Wikipédia)}
}
\newglossaryentry{pizza_teams}
{
  name=pizza teams,
  description={Une équipe polyvalente de taille 8 maximum (nombre de parts d'une pizza).}
}
\newglossaryentry{api}
{
  name=API,
  description={Une interface de programmation (souvent désignée par le terme API pour Application Programming Interface) est un ensemble normalisé de classes, de méthodes ou de fonctions qui sert de façade par laquelle un logiciel offre des services à d'autres logiciels. Elle est offerte par une bibliothèque logicielle ou un service web, le plus souvent accompagnée d'une description qui spécifie comment des programmes consommateurs peuvent se servir des fonctionnalités du programme fournisseur. (Wikipédia)}
}
\newglossaryentry{coding_dojo}
{
  name=coding dojo,
  description={Le coding dojo est une rencontre entre plusieurs personnes qui souhaitent travailler sur un défi de programmation de façon collective. Le défi peut être un problème algorithmique à résoudre ou un besoin à implémenter. Chaque coding dojo se concentre sur un sujet particulier, et représente l'objectif de la séance. Ce sujet doit permettre d'apprendre de façon collective sur le plan technique et sur la manière de réussir le défi. L'exercice peut être effectué entre personnes d'une même entreprise, d'une école ou encore venant d'horizons différents. (Wikipédia)}
}
\newglossaryentry{liens_hypertextes}
{
  name=liens hypertextes,
  description={Un hyperlien, ou lien hypertexte, ou lien web, ou simplement lien, est une référence dans un système hypertexte permettant de passer automatiquement d'un document consulté à un document lié. Les hyperliens sont notamment utilisés dans le World Wide Web pour permettre le passage d'une page Web à une autre à l'aide d'un clic. (Wikipédia)}
}
\newglossaryentry{http}
{
  name=HTTP,
  description={L'HyperText Transfer Protocol, plus connu sous l'abréviation HTTP — littéralement « protocole de transfert hypertexte » — est un protocole de communication client-serveur développé pour le World Wide Web. (Wikipédia)}
}
\newglossaryentry{html}
{
  name=HTML,
  description={L’Hypertext Markup Language, généralement abrégé HTML, est le format de données conçu pour représenter les pages web. C’est un langage de balisage permettant d’écrire de l’hypertexte, d’où son nom. HTML permet également de structurer s��mantiquement et de mettre en forme le contenu des pages, d’inclure des ressources multimédias dont des images, des formulaires de saisie, et des programmes informatiques. (Wikipédia)}
}
\newglossaryentry{xml}
{
  name=XML,
  description={L'Extensible Markup Language (XML, « langage à balise extensible » en français) est un langage informatique de balisage générique. (Wikipédia)}
}
\newglossaryentry{css}
{
  name=CSS,
  description={Les feuilles de style en cascade, généralement appelées CSS de l'anglais Cascading Style Sheets, forment un langage informatique qui décrit la présentation des documents HTML et XML. Les standards définissant CSS sont publiés par le World Wide Web Consortium (W3C). (Wikipédia)}
}
\newglossaryentry{responsive_web_design}
{
  name=responsive web design,
  description={Un site web adaptatif (anglais RWD pour responsive web design, conception de sites web adaptatifs selon l'OQLF1) est un site web dont la conception vise, grâce à différents principes et techniques, à offrir une expérience de consultation confortable même pour des supports différents. L'utilisateur peut ainsi consulter le même site web à travers une large gamme d'appareils (moniteurs d'ordinateur, smartphones, tablettes, TV, etc.) .(Wikipédia)}
}
\newglossaryentry{design_pattern}
{
  name=design pattern,
  description={un patron de conception (plus souvent appelé design pattern) est un arrangement caractéristique de modules, reconnu comme bonne pratique en réponse à un problème de conception d'un logiciel. Il décrit une solution standard, utilisable dans la conception de différents logiciels. (Wikipédia)}
}
\newglossaryentry{polyfills}
{
  name=polyfills,
  description={En programmation web, un polyfill est un ensemble de fonctions, le plus souvent écrites en Javascript ou en Flash, permettant de simuler sur un navigateur web ancien des fonctionnalités qui ne sont pas nativement disponibles. Par exemple, accéder à des fonctions HTML5 sur des navigateurs ne proposant pas ces fonctionnalités. (Wikipédia)}
}


%\newglossaryentry{ip}
%{
 % name=adresse I.P.,
 % plural=adresses I.P.,
 % description=Adresse qui identifie une machine sur un réseau informatique
%}

\newglossaryentry{scalable}
{
  name=scalabilité,
  description={On distingue deux types de scalabilité:
\begin{description}
\item[La scalabilité verticale] qui consiste à augmenter la capacité d'une application en améliorant la configuration matérielle sous-jacente (mémoire, processeur, disque dur…)
\item[La scalabilité horizontale] qui consiste à ajouter des machines pour améliorer les performances d'une application
\end{description}}
}

% Acronymes
%\newacronym[see={[Glossaire:]{orm_g}}]{orm}{ORM\glsadd{orm_g}}{Object Relational Mapping}
%\newacronym{poc}{POC}{Software Development Kit}
