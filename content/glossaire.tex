\newglossaryentry{nosql_g}
{
  name=NoSQL,
  description={Generalement des bases de données non-relationnelles}
}
\newglossaryentry{registrar}
{
  name=registraire de noms de domaines,
  description={Société accréditée par un registre pour créer de nouveaux noms de domaines. Il n'est plus nécessaire de passer par un prestataire intermédiaire pour acheter un nom de domaine. Le terme anglais est «Registrar»}
}
\newglossaryentry{whois}
{
  name=WhoIs,
  description={Contraction de l'Anglais Who Is.
     Ces données sont fournies par les registres de noms de domaine, et permettent d'en récupérer les données légales et techniques comme la date du dépôt, à qui appartient le domaine, les coordonnées des entités le gérant, etc.}
}
\newglossaryentry{ip}
{
  name=adresse I.P.,
  plural=adresses I.P.,
  description=Adresse qui identifie une machine sur un réseau informatique
}
\newglossaryentry{html}
{
  name=HTML,
  description=Code source d'une page internet
}
\newglossaryentry{full_text}
{
  name=recherche «full text»,
  plural=recherches «full text»,
  description={Recherche textuelle intelligente, capable de faire abstraction des conjugaisons, pluriels, synonymes…}
}
\newglossaryentry{framework}
{
  name=framework,
  description={Composant logiciel qui pose les fondations et grandes lignes d'un programme informatique, apportant de nombreuses fonctionnalités pré-implémentées génériques}
}
\newglossaryentry{acid_g}
{
  name=A.D.I.D,
  description={\url{https://fr.wikipedia.org/wiki/Propriétés_ACID}}
}
\newglossaryentry{snapshot_inc}
{
  name=snapshot incrémentaux,
  description={Capture à un instant donné de toute la base, en ne sauvegardant que les différences d'une version à une autre, de manière à minimiser l'espace nécessaire à la sauvegarde. Par restaurer une sauvegarde il est donc nécessaire d'en posséder toutes les précédentes}
}
\newglossaryentry{json_g}
{
  name=JSON,
  description={Format de stockage de données, \url{https://fr.wikipedia.org/wiki/JavaScript_Object_Notation}}
}
\newglossaryentry{master_node}
{
  name=nœud maître,
  description={Le nœud Maître, ou Master, est comparable à un chef d'orchestre, il est chargé de la coordination des nœuds}
}
\newglossaryentry{cookbook}
{
  name=recette Chef,
  description={Une recette décrit comment installer et configurer de manière automatique un logiciel sur une machine}
}
\newglossaryentry{autoscalling}
{
  name=autoscalling,
  description={Mécanisme permettant la création automatisée de machine pour répondre à un pic de charge. La création de ces instances est basée sur des mesures: mémoire utilisée, processeur utilisé… Lorsque la mesure dépasse le seuil configuré pendant un certain temps, de nouvelles instances peuvent être créées}
}
\newglossaryentry{versionning}
{
  name=gestionnaire de version,
  description={Logiciel permettant de sauvegarder et partager au sein d'une équipe de développement toutes les versions d'un code source d'un programme. Il est ainsi possible de savoir qui a fait quoi, quand, et de revenir à n'importe quelle version antérieure du code}
}
\newglossaryentry{fork}
{
  name=fork,
  description={Une nouvelle version parallèle, dans le même état}
}
\newglossaryentry{sql_g}
{
  name=SQL,
  description=Langage de requêtage traditionnel des bases de données relationnelles
}
\newglossaryentry{dao_g}
{
  name=DAO,
  description={Objet chargé de la récuparation des données depuis la base de données, et de leur conversion en objet utilisable par le programme}
}
\newglossaryentry{orm_g}
{
  name=ORM,
  description={Couche d'abstraction pour la communication avec la base de données. \url{https://fr.wikipedia.org/wiki/Mapping_objet-relationnel}}
}
\newglossaryentry{mitm}
{
  name=man in the middle,
  description={Attaque qui consiste à se placer entre deux interlocuteurs afin d'écouter/modifier la conversation}
}
\newglossaryentry{dns_g}
{
  name=DNS,
  description={Service qui permet de faire la transition entre un nom de domaine (e.g: exemple.com) en adresse utilisable par un programme (par exemple une adresse I.P)}
}
\newglossaryentry{ssl_g}
{
  name=SSL,
  description={Protocole de sécurisation pour les échanges de données sur Internet, permettant la sécurisation de toute communication TCP}
}
\newglossaryentry{scalable}
{
  name=scalabilité,
  description={On distingue deux types de scalabilité:
\begin{description}
\item[La scalabilité verticale] qui consiste à augmenter la capacité d'une application en améliorant la configuration matérielle sous-jacente (mémoire, processeur, disque dur…)
\item[La scalabilité horizontale] qui consiste à ajouter des machines pour améliorer les performances d'une application
\end{description}}
}
\newglossaryentry{s3_g}
{
    name=S3,
    description={Service Amazon de stockage de données}
}
\newglossaryentry{ec2_g}
{
    name=EC2,
    description={Service Amazon de location de machines virtuelles à l'heure}
}
\newglossaryentry{emr_g}
{
    name=EMR,
    description={Service Amazon de gestion de clusters de calcul Hadoop}
}
\newglossaryentry{hdfs_g}
{
    name=HDFS,
    description={\url{https://fr.wikipedia.org/wiki/Hadoop\#Hadoop_Distributed_File_System}}
}
\newglossaryentry{cluster}
{
    name=cluster,
    description={On parle de grappe de serveurs ou de ferme de calcul (\textit{computer cluster} en anglais) 
        pour désigner des techniques consistant à regrouper plusieurs ordinateurs indépendants appelés nœuds (\textit{node} en anglais),
    afin de permettre une gestion globale et de dépasser les limitations d'un ordinateur pour:
    \begin{itemize}
        \item augmenter la disponibilité,
        \item faciliter la montée en charge,
        \item permettre une répartition de la charge,
        \item augmenter les resources disponibles pour une application compatible. 
    \end{itemize}
}
}
\newglossaryentry{reverse_proxy}
{
    name=reverse proxy,
    description={
        Un proxy inverse (\textit{reverse proxy}) est un type de serveur, habituellement placé en frontal de serveurs web.
        Contrairement au serveur proxy qui permet à un utilisateur d'accéder au réseau Internet, le proxy inverse permet à un utilisateur d'Internet d'accéder à des serveurs internes
    }
}
% Acronymes
\newacronym[see={[Glossaire:]{orm_g}}]{orm}{ORM\glsadd{orm_g}}{Object Relational Mapping}
\newacronym[see={[Glossaire:]{dns_g}}]{dns}{DNS\glsadd{dns_g}}{Domain Name System}
\newacronym[see={[Glossaire:]{ssl_g}}]{ssl}{SSL\glsadd{ssl_g}}{Secure Socket Layer}
\newacronym[see={[Glossaire:]{nosql_g}}]{nosql}{NoSQL\glsadd{nosql_g}}{Not Only SQL}
\newacronym[see={[Glossaire:]{dao_g}}]{dao}{DAO\glsadd{dao_g}}{Data Access Object}
\newacronym[see={[Glossaire:]{sql_g}}]{sql}{SQL\glsadd{sql_g}}{Structured Query Language}
\newacronym[see={[Glossaire:]{json_g}}]{json}{JSON\glsadd{json_g}}{JavaScript Object Notation}
\newacronym[see={[Glossaire:]{acid_g}}]{acid}{A.C.I.D\glsadd{acid_g}}{Atomicité, Cohérence, Isolation, Durabilité}
\newacronym[see={[Glossaire:]{s3_g}}]{s3}{S3\glsadd{s3_g}}{Simple Storage Service}
\newacronym[see={[Glossaire:]{ec2_g}}]{ec2}{EC2\glsadd{ec2_g}}{Elastic Cloud Compute}
\newacronym[see={[Glossaire:]{emr_g}}]{emr}{EMR\glsadd{emr_g}}{Elastic MapReduce}
\newacronym[see={[Glossaire:]{hdfs_g}}]{hdfs}{HDFS\glsadd{hdfs_g}}{Hadoop Distributed File System}
\newacronym{aws}{AWS}{Amazon Web Services}
\newacronym{iam}{IAM}{Identity and Access Management}
\newacronym{api}{API}{Application Programming Interface}
\newacronym{http}{HTTP}{Hypertext Transfer Protocol}
\newacronym{sdk}{SDK}{Software Development Kit}
